\chapter{Class Polynomial}

\index{Polynomial}
\Polynomial\ is an object intended to analyze polynomials over GF(2). An object of type \Polynomial\ holds polynomial coefficients (as a list of positive integers) and a list of signs of those coefficients. Of course in case of GF(2) coefficient $x_i = -x_i$. However, negative coefficients make sense in case of some types of LFSRs, as \Polynomial\ objects are used to create other objects, of type of \Lfsr.

Below you can see an example of how to create a \Polynomial\ object representing the polynomial of $x^{16} + x^5 + x^2 + x^0$:
\begin{lstlisting}[language=Python]
		p1 = Polynomial ( [16, 5, 2, 0] )
		p2 = Polynomial ( 0b10000000000100101 )
		p2 = Polynomial ( 0x10025 )
\end{lstlisting}

\Polynomial\ class includes also a couple of static methods, especially useful to search for primitive polynomials and other ones discussed in the next part of this chapter.

\section{Polynomial object methods}

\cmd {Polynomial\_object} {\_\_str\_\_} {} {
	\Polynomial\ objects are convertible to strings.
}
\begin{lstlisting}[language=Python]
		p1 = Polynomial ( [16, 5, 2, 0] )
		print(p1)
		# >>> [16, 5, 2, 0]
\end{lstlisting}

\cmd {Polynomial\_object} {\_\_hash\_\_} {} {
	\Polynomial\ objects are hashable. Can be used as dictionary keys.:
}
\begin{lstlisting}[language=Python]
		p1 = Polynomial ( [16, 5, 2, 0] )
		d = {}
		d[p1] = "p1 value"
\end{lstlisting}

\cmd {Polynomial\_object} {copy} {} {
	Returns a deep copy of the \Polynomial\ object.
}
\begin{lstlisting}[language=Python]
		p1 = Polynomial ( [16, 5, 2, 0] )
		p2 = p1.copy()
		print(p1 == p2)
		# >>> True
\end{lstlisting}

\cmd {Polynomial\_object} {derivativeGF2} {} {
	Returns symbolic derivative Polynomial.
}
\begin{lstlisting}[language=Python]
		p1 = Polynomial ( [16, 15, 2, 1, 0] )
		print(p1.derivativeGF2())
		# >>> [14, 0]
		p2 = Polynomial ( [16, 14, 5, 2, 0] )
		p3 = p2.derivativeGF2()
		print(p3)
		# >>> [4]
\end{lstlisting}

\index{Polynomial!balancing}
\index{balancing of polynomial}
\label{polynomial:getbalancing}
\cmd {Polynomial\_object} {getBalancing} {} {
	Returns a difference between distances of furthest and closest coefficients.
}
\begin{lstlisting}[language=Python]
		p1 = Polynomial ( [16, 10, 2, 0] )
		#    distances:      6   8  2
		#    furthest:           8
		#    closest:               2
		#  furthest-closest:  8-2 = 6
		p1.getBalancing()
		# >>> 6
\end{lstlisting}

\cmd {Polynomial\_object} {getCoefficients} {} {
	Returns a reference to the sorted list of (unsigned) coefficients.
}
\begin{lstlisting}[language=Python]
		p1 = Polynomial ( [16, 2, -5, 0] )
		print(p1.getCoefficients())
		# >>> [16, 5, 2, 0]
\end{lstlisting}

\cmd {Polynomial\_object} {getCoefficientsCount} {} {
	Returns count of the \Polynomial\ object coefficients.
}
\begin{lstlisting}[language=Python]
		p1 = Polynomial ( [16, 5, 2, 0] )
		coeffscount1 = p1.getCoefficientsCount()
		print(coeffscount1)
		# >>> 4
\end{lstlisting}

\cmd {Polynomial\_object} {getDegree} {} {
	Returns degree of the \Polynomial\ object.
}
\begin{lstlisting}[language=Python]
		p1 = Polynomial ( [16, 5, 2, 0] )
		deg1 = p1.getDegree()
		print(deg1)
		# >>> 16
\end{lstlisting}

\cmd {Polynomial\_object} {getDifferentTapCount} {AnotherPolynomial} {
	Imagine, that the \texttt{Polynomial\_object} is a characteristic polynomial of a Ring Generator. Then this method compares the \texttt{Polynomial\_object} with another polynomial (also being a characteristic one of a Ring Generator) and returns a number of NOT matching taps. Tap direction (given by coefficient sign) does not matter.
}
\begin{lstlisting}[language=Python]
		p1 = Polynomial ( [16, -5, 2, 0] )
		p2 = Polynomial ( [16, 5, 2, 0] )
		p3 = Polynomial ( [16, 5, 1, 0] )
		p4 = Polynomial ( [16, 6, 0] )
		p1.getDifferentTapCount(p2)
		# >>> 0
		p1.getDifferentTapCount(p3)
		# >>> 1
		p1.getDifferentTapCount(p4)
		# >>> 2
		p4.getDifferentTapCount(p1)
		# >>> 1
\end{lstlisting}

\cmd {Polynomial\_object} {getMinDistance} {} {
	Returns a distance between closest Polynomial's coefficients.
}
\begin{lstlisting}[language=Python]
		p1 = Polynomial ( [16, 5, 2, 0] )
		#    distances:      11  3  2
		p1.getMinDistance()
		# >>> 2
\end{lstlisting}

\cmd {Polynomial\_object} {getReciprocal} {} {
	Returns a new, reciprocal Polynomial object.
}
\begin{lstlisting}[language=Python]
		p1 = Polynomial ( [16, 5, 2, 0] )
		print(p1.getReciprocal())
		# >>> [16, 14, 11, 0]
\end{lstlisting}

\cmd {Polynomial\_object} {getSignedCoefficients} {} {
	Returns sorted list of signed coefficients.
}
		\begin{lstlisting}[language=Python]
		p1 = Polynomial ( [16, 2, -5, 0] )
		print(p1.getSignedCoefficients())
		# >>> [16, -5, 2, 0]
\end{lstlisting}

\cmd {Polynomial\_object} {getSigns} {} {
	Returns signs of all sorted coefficients (as a list of 1s and -1s).
}
\begin{lstlisting}[language=Python]
		p1 = Polynomial ( [16, -5, 2, 0] )
		print(p1.getSigns())
		# >>> [1, -1, 1, 1]
\end{lstlisting}

\label{polynomial:islayoutfriendly}
\cmd {Polynomial\_object} {isLayoutFriendly} {} {
	Returns True if a Ring Generator, based on the Polynomial\_object, is layout friendly. It checks if the minimum distance between successive coefficients is at least 2.
}
\begin{lstlisting}[language=Python]
		p1 = Polynomial ( [16, 15, 2, 1, 0] )
		p1.isLayoutFriendly()
		# >>> False
		p2 = Polynomial ( [16, 14, 5, 2, 0] )
		p2.isLayoutFriendly()
		# >>> True
\end{lstlisting}

\cmd {Polynomial\_object} {isPrimitive} {} {
	Returns True if the given polynomial is primitive over GF(2). All coefficients are considered to be positive. Note, that the first call of this method may take more time than usual, because of prime dividers database loading.
	This methods bases on fast simulation of LFSRs described in \cite{lfsr:fastsim}.
}
\begin{lstlisting}[language=Python]
		p1 = Polynomial ( [16, 5, 2, 0] )
		p1.isPrimitive()
		# >>> False
		p2 = Polynomial ( [4, 1, 0] )
		p2.isPrimitive()
		# >>> True
\end{lstlisting}

\cmd {Polynomial\_object} {iterateThroughSigns} {} {
	This is generator method. Each time yields new \Polynomial\ object with other combinations of coefficient signs. Note, that the highest and lowest coefficients are untouched. All-positive and all-negative combinations are also not yielded.
}
\begin{lstlisting}[language=Python]
		p1 = Polynomial ( [16, -5, 2, 1, 0] )
		for pi in p1.iterateThroughSigns(): print(pi)
		# >>> [16, -5, 2, 1, 0]
		# >>> [16, 5, -2, 1, 0]
		# >>> [16, -5, -2, 1, 0]
		# >>> [16, 5, 2, -1, 0]
		# >>> [16, -5, 2, -1, 0]
		# >>> [16, 5, -2, -1, 0]
\end{lstlisting}

\cmd {Polynomial\_object} {nextPrimitive} {Silent=True} {
	Tries to find next polynomial which is primitive over GF(2). Returns True if found, otherwise returns False.
	if \texttt{Silent} argument is False, then searching process is shown in the terminal.
}
\begin{lstlisting}[language=Python]
		p1 = Polynomial ( [16, 15, 2, 1, 0] )
		p1.nextPrimitive()
		# >>> True
		print(p1)
		# >>> [16, 12, 3, 1, 0]
		p1.nextPrimitive()
		# >>> True
		print(p1)
		# >>> [16, 6, 4, 1, 0]
\end{lstlisting}

\cmd {Polynomial\_object} {printFullInfo} {} {
	Prints (also to the transcript in testcase mode) full info about the Polynomial\_object. See the example below:
}
\begin{lstlisting}[language=Python]
	p1 = Polynomial ( [16, 5, 2, 0] )
	p1.printFullInfo()
	# 
	# -------------------------
	# Polynomial  deg=16, bal=9
	# -------------------------
	# 
	# Degree            :  16
	# Coefficients count:  4
	# Hex with degree   :  10(2)25
	# Hex without degree:  25
	# Balancing         :  9
	# Is layout-friendly:  True
	# Coefficients      :  [16, 5, 2, 0]
\end{lstlisting}

\cmd {Polynomial\_object} {setStartingPointForIterator} {StartingPolynomial} {
	\Polynomial\ object may be used as generators, to iterate through all possible polynomials with respect to some requirements (see createPolynomial() method). This one is used to set starting point for iterator. See the example below.
	Note, that \texttt{StartingPolynomial} may be another \Polynomial\ object, or a list of coefficients. The starting polynomial is checked to have the same degree and coefficients count as the Polynomial\_object.
}
\begin{lstlisting}[language=Python]
		p1 = Polynomial ( [6,1,0] )
		for pi in p1: print(pi)
		# >>> [6, 1, 0]
		# >>> [6, 2, 0]
		# >>> [6, 3, 0]
		# >>> [6, 4, 0]
		# >>> [6, 5, 0]
		p1.setStartingPointForIterator( [6,3,0] )
		for pi in p1: print(pi)
		# >>> [6, 3, 0]
		# >>> [6, 4, 0]
		# >>> [6, 5, 0]
		p1.setStartingPointForIterator( Polynomial([6,4,0]) )
		for pi in p1: print(pi)
		# >>> [6, 4, 0]
		# >>> [6, 5, 0]
\end{lstlisting}

\cmd {Polynomial\_object} {toBitarray} {} {
	Returns a bitarray object representing the Polynomial.
}
\begin{lstlisting}[language=Python]
		p1 = Polynomial ( [16, 5, 2, 0] )
		p1.toBitarray()
		# >>> bitarray('10100100000000001')
\end{lstlisting}

\cmd {Polynomial\_object} {toHexString} {IncludeDegree=True, shorten=True} {
	Returns a string of hexadecimal characters describing the Polynomial\_object.
}
\begin{lstlisting}[language=Python]
		p1 = Polynomial ( [16, 5, 2, 0] )
		p1.toHexString()
		# >>> '10(2)25'
		p1.toHexString(IncludeDegree=False)
		# >>> '25'
		p1.toHexString(shorten=False)
		# >>> '10025'
\end{lstlisting}

\cmd {Polynomial\_object} {toInt} {} {
	Returns an integer representing the Polynomial object.
}
\begin{lstlisting}[language=Python]
		p1 = Polynomial ( [16, 5, 2, 0] )
		bin(p1.toInt())
		# >>> '0b10000000000100101'
\end{lstlisting}

\cmd {Polynomial\_object} {toMarkKassabStr} {} {
	Returns a string used by Mark Kassab's C++ code to add a polynomial to the internal database.
}
\begin{lstlisting}[language=Python]
		p1 = Polynomial ( [16, -5, 2, 0] )
		p1.toMarkKassabStr()
		# >>> 'add_polynomial(16, 5, 2, 0);'
\end{lstlisting}

\section{Static Polynomial methods}

\cmd {Polynomial} {checkPrimitives} {Candidates, n=0, Silent=True} {
	Takes a list of \Polynomial\ objects and returns a filtered list, containing only primitives. Uses multithreading.
	\begin{itemize}
		\item \texttt{Candidates} - list of polynomials to check,
		\item \texttt{n} - if \textit{n>0} then stops once n primitives found,
		\item \texttt{Silent} - if \texttt{False} then prints more info during searching. Otherwise (default) prints only a progress bar.
	\end{itemize}
}
\begin{lstlisting}[language=Python]
		Polynomial.checkPrimitives([ [4,1,0], [4,2,0], [4,3,0] ])
		# >>> [Polynomial([4, 3, 0]), Polynomial([4, 1, 0])]
\end{lstlisting}

\cmd {Polynomial} {createPolynomial} {PolynomialDegree, PolynomialCoefficientsCount, PolynomialBalancing=0, LayoutFriendly=False, MinDistance=0} {
	Returns a first \Polynomial\ object having a specified properties. Such object may be used then as iterator, to go through all polynomials having the same properties.
	\begin{itemize}
		\item \texttt{PolynomialDegree} - degree of polynomial,
		\item \texttt{PolynomialCoefficientsCount} - coefficients count, including degree and 0,
		\item \texttt{PolynomialBalancing} - maximum allowed balancing. '0' means 'no limit'. \\See \texttt{Polynomial.getBalancing} on page \pageref{polynomial:getbalancing} for more details,
		\item \texttt{LayoutFriendly} - wheather the polynomial must be layout-friendly or not. \\See \texttt{Polynomial.isLayoutFriendly} on page \pageref{polynomial:islayoutfriendly},
		\item \texttt{MinDistance} - minimum required distance between successive coefficients.
	\end{itemize}
}
\begin{lstlisting}[language=Python]
		p = Polynomial.createPolynomial(16, 4, MinDistance=4)
		p
		# >>> Polynomial([16, 8, 4, 0])
		for pi in p: print(pi)
		# >>> [16, 8, 4, 0]
		# >>> [16, 9, 4, 0]
		# >>> [16, 10, 4, 0]
		# >>> [16, 11, 4, 0]
		# ...
		# >>> [16, 12, 8, 0]
\end{lstlisting}

\cmd {Polynomial} {decodeUsingBerlekampMassey} {Sequence} {
	Simply realizes a Berlekamp-Massey algorithm to decode a characteristic polynomial basing on a sequence generated by an LFSR.
	\begin{itemize}
		\item \texttt{Sequence} - any iterable object, like bitarray, string and so on, or pointer to \Lfsr\ object.	\end{itemize}
}
\begin{lstlisting}[language=Python]
		# Decode a polynomial from a sequence:
		Polynomial.decodeUsingBerlekampMassey("110001")
		# >>> Polynomial([4, 3, 0])
		# Decode a characteristic polynomial of an LFSR:
		lfsr = Lfr([4,1,0], GALOIS)
		Polynomial.decodeUsingBerlekampMassey(lfsr)
		# >>> Polynomial([4, 1, 0])
\end{lstlisting}

\index{primitive polynomials}
\section{Static Polynomial methods to search for primitives}

Below are listed static \Polynomial\ method intended to search for primitive ones.
Each method has three versions:
\begin{enumerate}
	\item \texttt{list<*>Primitives} - returning a list of primitive polynomials,
	\item \texttt{print<*>Primitives} - printing found polynomials in a human-readable form,
	\item \texttt{first<*>Primitive} - returning a first found polynomial (or \texttt{None} if nothing found).
\end{enumerate}
All those methods have the same arguments, so that below are listed only those having a name starting from \texttt{list<*>}.
Consider also using \textit{interactive menu}, when working in interactive shell. It makes easier to configure below described searching methods (see page \pageref{menu}).

\cmd {Polynomial} {listAllPrimitivesHavingSpecifiedCoeffs} {CoefficientList, CoeffsCount=0,\\DontTouchBounds=True, OddOnly=True} {
	Takes a list of all available coefficients. Tries to search for primitive polynomials having coefficients included in the given list.
	\begin{itemize}
		\item \texttt{CoefficientList} - list of all available coefficients,
		\item \texttt{CoeffsCount} - specify here how many coefficients must have the found primitive polynomials (0 = 'any'),
		\item \texttt{DontTouchBounds} - if \texttt{True} then the highest and lowest coefficients (degree and 0) must always be included,
		\item \texttt{OddOnly} - limit searching only to such candidates having odd coefficient count.
	\end{itemize}
}
\begin{lstlisting}[language=Python]
		# Let's say we have a ring generator, having gated taps on such 
		# positions, so that the most rich polynomial implemented by thus
		# ring generator is:
		# [16, 14, 13, 12, 10, 9, 8, 5, 4, 2, 1, 0]
		# Let's find how many primitive polynomials may we implement
		# using such register:
		len(Polynomial.listAllPrimitivesHavingSpecifiedCoeffs(\
			[16, 14, 13, 12, 10, 9, 8, 5, 4, 2, 1, 0]))
		# >>> 49
		# Let's determine also how many pentanomials may we implement
		# using our register:
		len(Polynomial.listAllPrimitivesHavingSpecifiedCoeffs(\
			[16, 14, 13, 12, 10, 9, 8, 5, 4, 2, 1, 0], 5))
		# >>> 7
\end{lstlisting}

\cmd {Polynomial} {listDensePrimitives} {PolynomialDegree, m=0, Silent=True, StartingPolynomial=None} {
	Search for dense primitives \cite{lfsr:dense}.
	\begin{itemize}
		\item \texttt{PolynomialDegree} - degree of polynomial,
		\item \texttt{n} - if \textit{n>0} then stops once n primitives found,
		\item \texttt{Silent} - if \texttt{False} then prints more info during searching. Otherwise (default) prints only a progress bar,
		\item \texttt{StartingPolynomial} - you may specify a starting polynomial having the same properties as the ones passed to this method.
	\end{itemize}
}

\cmd {Polynomial} {listEveryNTapsPrimitives} {PolynomialDegree, EveryN, m=0, Silent=True, StartingPolynomial=None} {
	Search for well balanced (see page \pageref{polynomial:getbalancing}) polynomials having specified average distance between successive coefficients.
	\begin{itemize}
		\item \texttt{PolynomialDegree} - degree of polynomial,
		\item \texttt{EveryN} - average distance between successive coefficients.
		\item \texttt{n} - if \textit{n>0} then stops once n primitives found,
		\item \texttt{Silent} - if \texttt{False} then prints more info during searching. Otherwise (default) prints only a progress bar,
		\item \texttt{StartingPolynomial} - you may specify a starting polynomial having the same properties as the ones passed to this method.
	\end{itemize}
}

\cmd {Polynomial} {listHybridPrimitives} {PolynomialDegree, PolynomialCoefficientsCount,\\PolynomialBalancing=0, LayoutFriendly=False, MinDistance=0, m=0,\\NoResultsSkippingIteration=0, StartingPolynomial=None, MinNotMatchingTapsCount=0} {
	Search for polynomials which can create maximum length Hybrid LFSRa (see page \pageref{lfsr:hybrid}).
	\begin{itemize}
		\item \texttt{PolynomialDegree} - degree of polynomial,
		\item \texttt{PolynomialCoefficientsCount} - coefficients count, including degree and 0,
		\item \texttt{PolynomialBalancing} - maximum allowed balancing. '0' means 'no limit'. \\See \texttt{Polynomial.getBalancing} on page \pageref{polynomial:getbalancing} for more details,
		\item \texttt{LayoutFriendly} - wheather the polynomial must be layout-friendly or not. \\See \texttt{Polynomial.isLayoutFriendly} on page \pageref{polynomial:islayoutfriendly},
		\item \texttt{MinDistance} - minimum required distance between successive coefficients,
		\item \texttt{n} - if \textit{n>0} then stops once n primitives found,
		\item \texttt{Silent} - if \texttt{False} then prints more info during searching. Otherwise (default) prints only a progress bar,
		\item \texttt{NoResultsSkippingIteration} - if specified to be > 0 then it breaks the searching process if no result within the given number, successive iterations,
		\item \texttt{StartingPolynomial} - you may specify a starting polynomial having the same properties as the ones passed to this method,
		\item \texttt{MinNotMatchingTapsCount} - using this parameter you can specify at least how many taps must differ all found polynomials. In other words, if you specify minimum $k$ different taps and you get a result, so take any one of the found polynomials and compare with each one of the rest results - you can consider, than each time at least $k$ coefficients are different.
	\end{itemize}
}

\cmd {Polynomial} {listPrimitives} {PolynomialDegree, PolynomialCoefficientsCount, PolynomialBalancing=0, LayoutFriendly=False, MinDistance=0, m=0, Silent=True, MaxSetSize=10000, ExcludeList=[],\\ReturnAlsoAllCandidaes=False, FilteringCallback=None, NoResultsSkippingIteration=0,\\StartingPolynomial=None} {
	The most general method to search for primitive polynomials.
	\begin{itemize}
		\item \texttt{PolynomialDegree} - degree of polynomial,
		\item \texttt{PolynomialCoefficientsCount} - coefficients count, including degree and 0,
		\item \texttt{PolynomialBalancing} - maximum allowed balancing. '0' means 'no limit'. \\See \texttt{Polynomial.getBalancing} on page \pageref{polynomial:getbalancing} for more details,
		\item \texttt{LayoutFriendly} - wheather the polynomial must be layout-friendly or not. \\See \texttt{Polynomial.isLayoutFriendly} on page \pageref{polynomial:islayoutfriendly},
		\item \texttt{MinDistance} - minimum required distance between successive coefficients,
		\item \texttt{n} - if \textit{n>0} then stops once n primitives found,
		\item \texttt{Silent} - if \texttt{False} then prints more info during searching. Otherwise (default) prints only a progress bar,
		\item \texttt{MaxSetSize} - how many polynomials should be tested in parallel in one iteration,
		\item \texttt{ExcludeList} - polynomials included in this list will e skipped from checking,
		\item \texttt{ReturnAlsoAllCandidaes} - if \texttt{True} then as a result this method will return list of 2 lists: \\\texttt{[ [primitive\_polys], [all\_checked\_polys] ]},
		\item \texttt{FilteringCallback} - if specified, then will be called for each candidate polynomial:\\\texttt{callback(candidate\_polynomial)}\\and should return \texttt{True} if the given polynomial should be checked, otherwise the given polynomial will be skipped from checking,
		\item \texttt{NoResultsSkippingIteration} - if specified to be > 0 then it breaks the searching process if no result within the given number, successive iterations,
		\item \texttt{StartingPolynomial} - you may specify a starting polynomial having the same properties as the ones passed to this method.
	\end{itemize}
}

\cmd {Polynomial} {listTapsFromTheLeftPrimitives} {PolynomialDegree, PolynomialCoefficientsCount, MaxDIstance=3, m=0, Silent=True} {
	Search for such primitive polynomials having coefficients grouped near the left corner, like \texttt{[10,9,8,7,0]}.
	\begin{itemize}
		\item \texttt{PolynomialDegree} - degree of polynomial,
		\item \texttt{PolynomialCoefficientsCount} - coefficients count, including degree and 0,
		\item \texttt{MaxDistance} - maximum required distance between successive coefficients grouped near the left corner,
		\item \texttt{n} - if \textit{n>0} then stops once n primitives found,
		\item \texttt{Silent} - if \texttt{False} then prints more info during searching. Otherwise (default) prints only a progress bar.
	\end{itemize}
}

\cmd {Polynomial} {listTigerPrimitives} {PolynomialDegree, PolynomialCoefficientsCount,\\PolynomialBalancing=0, LayoutFriendly=False, MinDistance=0, m=0,\\NoResultsSkippingIteration=0, StartingPolynomial=None, MinNotMatchingTapsCount=0} {
	Search for polynomials which can create maximum length Tiger LFSRa (see page \pageref{lfsr:tiger}).
	\begin{itemize}
		\item \texttt{PolynomialDegree} - degree of polynomial,
		\item \texttt{PolynomialCoefficientsCount} - coefficients count, including degree and 0,
		\item \texttt{PolynomialBalancing} - maximum allowed balancing. '0' means 'no limit'. \\See \texttt{Polynomial.getBalancing} on page \pageref{polynomial:getbalancing} for more details,
		\item \texttt{LayoutFriendly} - wheather the polynomial must be layout-friendly or not. \\See \texttt{Polynomial.isLayoutFriendly} on page \pageref{polynomial:islayoutfriendly},
		\item \texttt{MinDistance} - minimum required distance between successive coefficients,
		\item \texttt{n} - if \textit{n>0} then stops once n primitives found,
		\item \texttt{Silent} - if \texttt{False} then prints more info during searching. Otherwise (default) prints only a progress bar,
		\item \texttt{NoResultsSkippingIteration} - if specified to be > 0 then it breaks the searching process if no result within the given number, successive iterations,
		\item \texttt{StartingPolynomial} - you may specify a starting polynomial having the same properties as the ones passed to this method,
		\item \texttt{MinNotMatchingTapsCount} - using this parameter you can specify at least how many taps must differ all found polynomials. In other words, if you specify minimum $k$ different taps and you get a result, so take any one of the found polynomials and compare with each one of the rest results - you can consider, than each time at least $k$ coefficients are different.
	\end{itemize}
}
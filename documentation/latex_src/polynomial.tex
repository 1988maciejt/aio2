\chapter{Class Polynomial}

\index{Polynomial}
\Polynomial\ is an object intended to analyze polynomials over GF(2). An object of type \Polynomial\ holds polynomial coefficients (as a list of positive integers) and a list of signs of those coefficients. Of course in case of GF(2) coefficient $x_i = -x_i$. However, negative coefficients make sense in case of some types of LFSRs, as \Polynomial\ objects are used to create other objects, of type of \Lfsr.

Below you can see an example of how to create a \Polynomial\ object representing the polynomial $x^{16} + x^5 + x^2 + x^0$:
\begin{lstlisting}[language=Python]
	p1 = Polynomial ( [16, 5, 2, 0] )
	p2 = Polynomial ( 0b10000000000100101 )
	p2 = Polynomial ( 0x10025 )
\end{lstlisting}

\Polynomial\ class includes also a couple of static methods, especially useful to search for primitive polynomials and other ones discussed in the next part of this chapter.

\section{Polynomial object methods}

\cmd{str(<Polynomial>)}{
	\Polynomial\ objects are convertible to strings.
}
\begin{lstlisting}[language=Python]
	p1 = Polynomial ( [16, 5, 2, 0] )
	print(p1)
	# >>> [16, 5, 2, 0]
\end{lstlisting}

\cmd{hash(<Polynomial>)}{
	\Polynomial\ objects are hashable. Can be used as a dictionary keys.:
}
\begin{lstlisting}[language=Python]
	p1 = Polynomial ( [16, 5, 2, 0] )
	d = {}
	d[p1] = "p1 value"
\end{lstlisting}

\cmd{<Polynomial>.getCoefficients()}{
	Returns a reference to coefficients list of the \Polynomial\ object.
}
\begin{lstlisting}[language=Python]
	p1 = Polynomial ( [16, 5, 2, 0] )
	coeffs1 = p1.getCoefficients()
	print(coeffs1)
	# >>> [16, 5, 2, 0]
	coeffs.remove(0)
	print(coeffs1)
	# >>> [16, 5, 2]
	print(p1)
	# >>> [16, 5, 2]
\end{lstlisting}

\cmd{<Polynomial>.getCoefficientsCount()}{
	Returns count of the \Polynomial\ object coefficients.
}
\begin{lstlisting}[language=Python]
	p1 = Polynomial ( [16, 5, 2, 0] )
	coeffscount1 = p1.getCoefficientsCount()
	print(coeffscount1)
	# >>> 4
\end{lstlisting}

\cmd{<Polynomial>.getDegree()}{
	Returns degree of the \Polynomial\ object.
}
\begin{lstlisting}[language=Python]
	p1 = Polynomial ( [16, 5, 2, 0] )
	deg1 = p1.getDegree()
	print(deg1)
	# >>> 16
\end{lstlisting}

\section{Static Polynomial methods}
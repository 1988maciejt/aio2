\chapter{Class ProgrammableLfsr}
\index{Programmable LFSR}

\ProgrammableLfsr\ is a class intended to make easy to design and analyse any kind of LFSR with programmable feedback taps.  To create a \ProgrammableLfsr\ object one need to specify a list of configurable taps. Each configurable tap is described by a dictionary. Each key of such dictionary is a custom name, like "tapOn", or just a number - does not matter - and is used to make analysis reports readable. The corresponding value is a standard tap definition: \texttt{[<source\_FF\_index>, <destination\_FF\_index>]}. Consider some examples of configurable tap dictionaries:
\label{configurabletapsdictionary}
\begin{itemize}
	\item \texttt{\{'Tap1':[3 6]\}} - mandatory (constant) tap from output of \texttt{FF[3]} to a XOR gate at input of \texttt{FF[6]}
	\item \texttt{\{'Tap1\_OFF':None, 'Tap1\_ON':[3,6]\}} - gated tap from output of \texttt{FF[3]} to a XOR gate at input of \texttt{FF[6]}. Consider the value of the first key: \texttt{None}, which means \textit{no connection}
	\item \texttt{\{'Sw1':[3,6], 'Sw2':[3,8]\}} - switched tap: from output of \texttt{FF[3]} to a XOR gate at input of \texttt{FF[6]}, or from output of \texttt{FF[3]} to a XOR gate at input of \texttt{FF[8]}
	\item \texttt{\{'Sw0':None, 'Sw1':[3,6], 'Sw2':[3,8]\}} - like above, but this time there is a way to disable (no connection)
	\item \texttt{\{'Sw1':[3,6], 'Sw2':[4,8]\}} - switched tap: from output of \texttt{FF[3]} to a XOR gate at input of \texttt{FF[6]}, or from output of \texttt{FF[8]} to a XOR gate at input of \texttt{FF[8]}
\end{itemize}
Consider, that there is no required to make common source or common destination anywhere. See also \ProgrammableLfsrConfig\ utility class on page \pageref{programmablelefsrconfig}

\section{ProgrammableLfsr object methods}

\cmd {ProgrammableLfsr\_object} {\_\_init\_\_} {SizeOrProgrammableLfsrConfiguration, TapsList=[],\\OperateOnTapsOnly=False} {
	\ProgrammableLfsr\ object initializer.
	\begin{itemize}
		\item \texttt{SizeOrProgrammableLfsrConfiguration} - size of the LFSR or a reference to \ProgrammableLfsrConfig\ utility object
		\item \texttt{TapsList} - list of Configurable Taps Dictionaries (see description on page \pageref{configurabletapsdictionary})
		\item \texttt{OperateOnTapsOnly} - as a result of analysis you may get a list of \Lfsr\ objects. However, to save the memory, by setting this parameter to \texttt{True}, you may force to operate on tap lists only, so that tap lists will be returned instead of lists of \Lfsr\ objects.
	\end{itemize}
}
\begin{lstlisting}[language=Python]
	# we need a 8-bit LFSR having:
	# - a mandatory tap from output of FF[7] to a XOR at input of FF[1],
	# - a gated tap from output of FF[5] to a XOR at input of FF[3], 
	ProgLfsr1 = ProgrammableLfsr(8, [
		{'tap1':[7,1]},
		{'tap2_off':None, 'tap2_on':[5,3]}
	])
\end{lstlisting}

\cmd {ProgrammableLfsr\_object} {clear} {} {
	It simply deletes results of Programmable Lfsr analysis to save the memory
}

\cmd {ProgrammableLfsr\_object} {getAllPosssibleTaps} {} {
	Returns a list containing all connections (taps) possible to implement by the \ProgrammableLfsr\ object. 
	See the example on page \pageref{programmablelfsrexample}
}

\cmd {ProgrammableLfsr\_object} {getSize} {} {
	Returns a size (flip-flop count)of the Programmable Lfsr object.
}

\cmd {ProgrammableLfsr\_object} {getPolynomialsAndLfsrsDictionary} {Optimization=False} {
	Returns a dictionary, containing all primitive polynomials the \ProgrammableLfsr\ object can implement and corresponding \Lfsr' objects (or tap lists). If \texttt{Optimization} is set to \texttt{True}, then it tries to optimize all configurations of the programmable LFSR so that as less taps are used, as possible.
	See the example on page \pageref{programmablelfsrexample}
}

\cmd {ProgrammableLfsr\_object} {getLfsrs} {Optimization=True} {
	Returns a list containing all maximum Lfsrs implemented by the \ProgrammableLfsr\ object. If \texttt{Optimization} is set to \texttt{True}, then it tries to optimize all configurations of the programmable LFSR so that as less taps are used, as possible.
	See the example on page \pageref{programmablelfsrexample}
}

\cmd {ProgrammableLfsr\_object} {getPolynomials} {} {
	Returns a list containing all primitive polynomials implemented by the \ProgrammableLfsr\ object.
	See the example on page \pageref{programmablelfsrexample}
}

\cmd {ProgrammableLfsr\_object} {getUsedTaps} {} {
	Returns a list containing used connections (taps) after optimization. 
	See the example on page \pageref{programmablelfsrexample}
}

\cmd {ProgrammableLfsr\_object} {getUnusedTaps} {} {
	Returns a list containing unused connections (taps) after optimization. Removing those connections will not affect the primitive polynomials list implemented by this \ProgrammableLfsr\ object.
	See the example on page \pageref{programmablelfsrexample}
}

\cmd {ProgrammableLfsr\_object} {toVerilog} {ModuleName, InjectorIndexesList=[]} {
	Returns a string containing Verilog description of the \ProgrammableLfsr\ object.
	\begin{itemize}
		\item \texttt{ModuleName} - name of the Verilog module,
		\item \texttt{InjectorIndexesList} - a list containing indexes of flip-flops at which input injectors have to be placed.
	\end{itemize}
	See the example on page \pageref{programmablelfsrexample}
}

\label{programmablelfsrexample}
Example of Programmable Lfsr analysis:
\begin{lstlisting}[language=Python]
# let's examine a 8-bit LFSR having:
# - a gated tap from output of FF[6] to a XOR at input of FF[0], 
# - a gated tap from output of FF[5] to a XOR at input of FF[0], 
# - a switchable tap tap from output of FF[4] to a XOR at input of FF[3]
#     of from output of FF[3] to a XOR at input of FF[4]
#     with no-connection (disable) option
# - a switchable tap tap from output of FF[2] to a XOR at input of FF[3]
#     of from output of FF[4] to a XOR at input of FF[2]
#     with no-connection (disable) option
plfsr = ProgrammableLfsr(8, [
{'6-0_off': None, '6-0_on': [6, 0]},
{'5-0_off': None, '5-0_on': [5, 0]},
{'None': None, '4-3': [4, 3], '3-4': [3, 4]},
{'None': None, '2-3': [2, 3], '4-2': [4, 2]}
], OperateOnTapsOnly = True)
plfsr.getPolynomialsAndLfsrsDictionary()
# >>> {Polynomial([8, 5, 3, 2, 0]): [[[6, 0], [3, 4]], [[6, 0], [2, 3]]]}
plfsr.getPolynomialsAndLfsrsDictionary(True)
# >>> {Polynomial([8, 5, 3, 2, 0]): [[6, 0], [2, 3]]}
# As we can see, such Programmable Lfsr can implement 2 maximum Lfsrs,
# but only 1 primitive polynomial.
# We can even get a simple list of primitive polynomials implemented
# by the 'plfsr':
plfsr.getPolynomials()
# >>> [Polynomial([8, 5, 3, 2, 0])]
# ...and a simple list of all possible Lfsrs (tap lists in our case).
# So without optimization (all possible, maximum ones):
plfsr.getLfsrs(False)
# >>> [[[6, 0], [3, 4]], [[6, 0], [2, 3]]]
# ...and with optimization:
plfsr.getLfsrs(True)
# >>> [[[6, 0], [3, 4]]]
# DO you know all possible connections? Let's check it:
plfsr.getAllPosssibleTaps()
# >>> [[6, 0], [5, 0], [4, 3], [3, 4], [2, 3], [4, 2]]
# Now determine, which connections are used:
plfsr..getUsedTaps()
# >>> [[6, 0], [3, 4]]
# ...so it seems only 2 taps are required. So which ones are unused?
plfsr.getUnusedTaps()
# >>> [[5, 0], [4, 3], [2, 3], [4, 2]]
# Finally, let's generate a verilog describingout Programmable Lfsr.
# In addition assume we have injectors at FF[2,1,0]:
print(plfsr.toVerilog("MyProgrammableLfsr", [2,1,0]))
# module MyProgrammableLfsr (
# 	input wire clk,
# 	input wire enable,
# 	input wire reset,
# 	input wire [2:0] injectors,
# 	output reg [7:0] O
# );
# 
# reg T0;
# reg T0_0;
# reg T1;
# reg T1_0;
# reg T2;
# reg T2_3;
# reg T2_4;
# reg T3;
# reg T3_3;
# reg T3_2;
# 
# 
# always @ (*) begin
# 	T0 <= 1'b0;
# 	if (config[0] == 1'd1)) begin
# 		T0 <= O[6];
# 	end
# end
# 
# always @ (*) begin
# 	T1 <= 1'b0;
# 	if (config[1] == 1'd1)) begin
# 		T1 <= O[5];
# 	end
# end

# always @ (*) begin
# 	T2 <= 1'b0;
# 	if (config[3:2] == 2'd1)) begin
# 		T2 <= O[4];
# 	end
# 	if (config[3:2] == 2'd2)) begin
# 		T2 <= O[3];
# 	end
# end
# 
# always @ (*) begin
# 	T3 <= 1'b0;
# 	if (config[5:4] == 2'd1)) begin
# 		T3 <= O[2];
# 	end
# 	if (config[5:4] == 2'd2)) begin
# 		T3 <= O[4];
# 	end
# end
# 
# always @ (*) begin
# 	T0_0 <= 1'b0;
# 	if ((config[0:0] == 1'd1)) begin
# 		T0_0 <= T0;
# 	end
# end
# 
# always @ (*) begin
# 	T1_0 <= 1'b0;
# 	if ((config[1:1] == 1'd1)) begin
# 		T1_0 <= T1;
# 	end
# end
# 
# always @ (*) begin
# 	T2_3 <= 1'b0;
# 	if ((config[3:2] == 2'd1)) begin
# 		T2_3 <= T2;
# 	end
# end
# 
# always @ (*) begin
# 	T2_4 <= 1'b0;
# 	if ((config[3:2] == 2'd2)) begin
# 		T2_4 <= T2;
# 	end
# end
# 
# always @ (*) begin
# 	T3_3 <= 1'b0;
# 	if ((config[5:4] == 2'd1)) begin
# 		T3_3 <= T3;
# 	end
# end
# 
# always @ (*) begin
# 	T3_2 <= 1'b0;
# 	if ((config[5:4] == 2'd2)) begin
# 		T3_2 <= T3;
# 	end
# end
# 
# always @ (posedge clk or posedge reset) begin
# 	if (reset) begin
# 		O <= 8'd0;
# 	end else begin
# 		if (enable) begin
# 			O[0] <= O[1] ^ T0_0 ^ T1_0 ^ injectors[2];
# 			O[1] <= O[2] ^ injectors[1];
# 			O[2] <= O[3] ^ T3_2 ^ injectors[0];
# 			O[3] <= O[4] ^ T2_3 ^ T3_3;
# 			O[4] <= O[5] ^ T2_4;
# 			O[5] <= O[6];
# 			O[6] <= O[7];
# 			O[7] <= O[0];
# 		end
# 	end
# end
# 
# endmodule
\end{lstlisting}

\section{ProgrammableLfsrConfig utility class}
\label{programmablelefsrconfig}

\ProgrammableLfsrConfig\ is a utility class making easier to create a configuration list for \ProgrammableLfsr. Consider the example from page \pageref{programmablelfsrexample}. Let's try to make the same \ProgrammableLfsr\ object but this time using the \ProgrammableLfsrConfig\ utility class:
\begin{lstlisting}[language=Python]
# let's examine a 8-bit LFSR having:
# - a gated tap from output of FF[6] to a XOR at input of FF[0], 
# - a gated tap from output of FF[5] to a XOR at input of FF[0], 
# - a switchable tap tap from output of FF[4] to a XOR at input of FF[3]
#     of from output of FF[3] to a XOR at input of FF[4]
#     with no-connection (disable) option
# - a switchable tap tap from output of FF[2] to a XOR at input of FF[3]
#     of from output of FF[4] to a XOR at input of FF[2]
#     with no-connection (disable) option
config = ProgrammableLfsrConfiguration(8)
config.addGated(6,0)
# >>> {'6-0_off': None, '6-0_on': [6, 0]}
config.addGated(5,0)
# >>> {'5-0_off': None, '5-0_on': [5, 0]}
config.addSwitched(None,[4,3],[3,4])
# >>> {'None': None, '4-3': [4, 3], '3-4': [3, 4]}
config.addSwitched(None,[2,3],[4,2])
# >>> {'None': None, '2-3': [2, 3], '4-2': [4, 2]}
plfsr = ProgrammableLfsr(config, OperateOnTapsOnly=True)
\end{lstlisting}

Consider also a way to create all-combinations switch:

\cmd {ProgrammableLfsr\_object} {addAllCombinationsSwitch} {SourceList, DestinationList, IncludeNone=False} {
	\begin{itemize}
		\item \texttt{SourceList} - list of source FFs indexes
		\item \texttt{DestinationList} - list of destination FFs indexes
		\item \texttt{IncludeNone} - if \texttt{True}, then \texttt{None} will also be included in the resultant taps switch
	\end{itemize}
}
\begin{lstlisting}[language=Python]
	config = ProgrammableLfsrConfiguration(8)
	config.addAllCombinationsSwitch([1,2], [4,5])
	# >>> {'1-4': [1, 4], '1-5': [1, 5], '2-4': [2, 4], '2-5': [2, 5]}
	config.addAllCombinationsSwitch([1,2], [4,5], True)
	# >>> {'None': None, '1-4': [1, 4], '1-5': [1, 5], '2-4': [2, 4], '2-5': [2, 5]}
\end{lstlisting}

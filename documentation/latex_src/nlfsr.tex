\chapter{Class Nlfsr}
\index{Nonlinear Feedback Shift Register}
\index{NLFSR}

NLFSR (\textit{Non-Linear Feedback Shift Register}) is very similar to LFSR (\textit{Linear Feedback Shift Register}) with the only difference: feedback taps may be results of a non-linear function of register's flip-flops. Consider the example from Fig. \ref{nlfsr.example}. We can see there a 10-bit register having two taps: \texttt{from FF[4] to a XOR at input of FF[5]} and \textit{from (FF[7] AND FF[3]) to a XOR at input of FF[0])}. This may be strictly implemented using \Nlfsr\ object by:
\begin{lstlisting}[language=Python]
	MyNLFSR = Nlfsr(10, [ [5, [4]], [0, [3,7]] ])
	#        size >|--|<
	#                   |<------- taps ------->|
	# tap 1: [5, [4]]   : FF[4] to input of FF[5]
	# tap 2: [0. [3,7]] : (FF[3] AND FF[7]) to input of FF[0]
\end{lstlisting}

\begin{figure}[h]
	\centering
	\scalebox{.75}{\newcommand{\drawff}[2]{		
	\fill[black!5!white] 		($#1-(0.5, 0.5)$)	rectangle	($#1+(0.5, 0.5)$);
	\draw[thick] 		($#1-(0.5, 0.5)$)	rectangle	($#1+(0.5, 0.5)$);
	\node[] at ($#1+(0.0, 0.0)$) {\large#2};
}
\newcommand{\drawring}[2]{
	\coordinate (A) at ($#1-(1.0,0.0)$);
	\coordinate (A2) at ($(A)+(0.5,0.0)$);
	\coordinate (B) at ($#2+(1.0,0.0)$);
	\coordinate (B2) at ($(B)-(0.5,0.0)$);
	\draw (A) rectangle (B);
	\draw[-latex] (A) -- (A2);
	\draw[-latex] (B) -- (B2);
}
\newcommand{\drawringconnectorup}[3]{
	\coordinate (A) at ($#1+(0.0,1.0)$);
	\coordinate (B) at ($#2-(0.0,1)$);
	\coordinate (C) at ($0.3*(B)+0.7*(A)+(0.0,0.2)$);
	\draw[-latex] #1 -- (A) -- (B) -- ($#2-(0.0,0.2)$);
	\fill[black]  ($#1+(0.0, 0.0)$) circle (0.07);
	\fill[black!5!white]  ($#2+(0.0, 0.0)$) circle (0.20);
	\draw[thick]  ($#2+(0.0, 0.0)$) circle (0.2);
	\draw[thick]  ($#2+(0.0, 0.0)-(0.2,0.0)$) -- ($#2+(0.0, 0.0)+(0.2,0.0)$);
	\draw[thick]  ($#2+(0.0, 0.0)-(0.0,0.2)$) -- ($#2+(0.0, 0.0)+(0.0,0.2)$);
	\node[anchor=west] at (C) {\Large#3};
}
\newcommand{\drawringconnectordown}[3]{
	\coordinate (A) at ($#1-(0.0,1.0)$);
	\coordinate (B) at ($#2+(0.0,1)$);
	\coordinate (C) at ($0.3*(A)+0.7*(B)+(0.0,0.2)$);
	\draw[-latex] #1 -- (A) -- (B) -- ($#2+(0.0,0.2)$);
	\fill[black]  ($#1+(0.0, 0.0)$) circle (0.07);
	\fill[black!5!white]  ($#2+(0.0, 0.0)$) circle (0.20);
	\draw[thick]  ($#2+(0.0, 0.0)$) circle (0.2);
	\draw[thick]  ($#2+(0.0, 0.0)-(0.2,0.0)$) -- ($#2+(0.0, 0.0)+(0.2,0.0)$);
	\draw[thick]  ($#2+(0.0, 0.0)-(0.0,0.2)$) -- ($#2+(0.0, 0.0)+(0.0,0.2)$);
	\node[anchor=west] at (C) {\Large#3};
}
\newcommand{\drawxor}[1]{
	\draw[thick]  ($#1+(0.0, 0.0)$) circle (0.2);
	\draw[thick]  ($#1+(0.0, 0.0)-(0.2,0.0)$) -- ($#1+(0.0, 0.0)+(0.2,0.0)$);
	\draw[thick]  ($#1+(0.0, 0.0)-(0.0,0.2)$) -- ($#1+(0.0, 0.0)+(0.0,0.2)$);
}
\begin{tikzpicture}
	\drawring {(0,0)} {(8,3)};
	
	\drawff {(0,0)} {4}
	\drawff {(0,3)} {5}
	
	\drawff {(2,0)} {3}
	\drawff {(2,3)} {6}
	
	\drawff {(4,0)} {2}
	\drawff {(4,3)} {7}	
	
	\drawff {(6,0)} {1}
	\drawff {(6,3)} {8}
	
	\drawff {(8,0)} {0}
	\drawff {(8,3)} {9}
	
	\draw[thick] (4.5, 2) -- (4, 2) -- (4, 1) -- (4.5, 1) arc [start angle=-90, delta angle=180, radius=0.5] -- (4.5, 2);
	\draw (3, 3) -- (3, 1.75) -- (4, 1.75);
	\draw (3, 0) -- (3, 1.25) -- (4, 1.25);
	\draw[latex-] (7, 0.2) -- (7, 1.5) -- (5, 1.5);
	\fill[black]  (3,3) circle (0.07);
	\fill[black]  (3,0) circle (0.07);
	\drawxor{(7,0)}
	\drawringconnectorup   {(1,0)} {(1,3)} {}
\end{tikzpicture}}
	\caption{Example of Nonlinear Feedback Shift Register.}
	\label{nlfsr.example}
\end{figure}

\label{nlfsr:taps}
If is worth nothing, that tap description of \Nlfsr\ is different than tap description of \Lfsr. The difference is, that while each tap is still defined by a list, but the first item is not a source, but destination tap and the second item is a list of ANDed sources. So, each tap of \Nlfsr\ looks like: \\
\texttt{[ <destination\_FF>, [<source\_FF\_1>, <source\_FF\_2>, ...] ]}.

As you can see, source values may only be ANDed. However, there is still a way to make a tap implementing other nonlinear function. All indexes given in tap list may also be negative numbers. If a source index is negative, then it considers NOT(source). If a destination index is negative, then it considers NAND instead of AND. See example taps description:
\begin{itemize}
	\item \texttt{[2, [-3,4]]} - AND(NOT(FF[3]), FF[4]) to a XOR at input of FF[2]
	\item \texttt{[-2, [3,4]]} - NAND(FF[3], FF[4]) to a XOR at input of FF[2]
	\item \texttt{[-2, [-3,-4]]} - NAND(NOT(FF[3]), NOT(FF[4])) = OR(FF[3], FF[4]) to a XOR at input of FF[2]
\end{itemize}

Keep in mind, that there is not any \textit{fast simulation} method for NLFSRs. The only way to determine the period of NLFSR is simply simulate it step-by-step. To make it as fast as possible, he \ShellName\ implements a C++ based subroutine.

\section{Nlfsr object methods}

Not all \Nlfsr\ object methods will be described. \Nlfsr\ class bases on \Lfsr\ class, so almost all methods are inherited from LFSR.

\cmd {Nlfsr\_object} {\_\_init\_\_} {Size, Config=[]} {
	Object initializer.
	\begin{itemize}
		\item \texttt{Size} - size (flip-flop count) of the NLFSR,
		\item \texttt{Config} - list of taps definition (see page \pageref{nlfsr:taps})
	\end{itemize}
}

\cmd {Nlfsr\_object} {\_\_str\_\_} {} {
	It returns a string containing binary value of the \Nlfsr\ object (left MSb). See also \texttt{Lfsr.\_\_str\_\_} on page \pageref{lfsr:str}
}

\cmd {Nlfsr\_object} {getArchitecture} {} {
	Returns a string containing human-readable description of taps.
}
\begin{lstlisting}[language=Python]
	MyNLFSR = Nlfsr(10, [ [5, [4]], [0, [3,7]] ])
	print(MyNLFSR.getArchitecture())
	# >>>  0 <=  3 AND 7
	# >>>  5 <=  4
\end{lstlisting}

\cmd {Nlfsr\_object} {getFullInfo} {Header=False} {
	Returns a string containing some information about NLFSR. Dependently on \texttt{Header} value, the first line is a header (see the example below). Next lines are human-readable description of taps. Last four lines are simplified equations - update functions - of the NLFSR. First update function - NLFSR as is; the second - complementary; third - reversed, and finally - reversed-complementary.
}
\begin{lstlisting}[language=Python]
	MyNLFSR = Nlfsr(10, [ [5, [4]], [0, [3,7]] ])
	print(MyNLFSR.getFullInfo(True))
	# >>>  10-bit NLFSRs  and taps list and equations:
	# >>>  0 <=  3 AND 7
	# >>>  5 <=  4
	# >>> 2, (4, 8)
	# >>> C  2, 4, 8, (4, 8)
	# >>> R 8, (6, 2)
	# >>> CR 2, 6, 8, (6, 2)
\end{lstlisting}

\cmd {Nlfsr\_object} {next} {steps=1} {
	See \texttt{Lfsr.next} on page \pageref{lfsr:next}.
}

\cmd {Nlfsr\_object} {printFullInfo} {} {
	Prints the full information (without header) about the NLFSR. Does the same as \textit{Aio.print(NlfsrObject.getFullInfo(False))}.
}
\begin{lstlisting}[language=Python]
	MyNLFSR = Nlfsr(10, [ [5, [4]], [0, [3,7]] ])
	MyNLFSR.printFullInfo()
	# >>>  0 <=  3 AND 7
	# >>>  5 <=  4
	# >>> 2, (4, 8)
	# >>> C  2, 4, 8, (4, 8)
	# >>> R 8, (6, 2)
	# >>> CR 2, 6, 8, (6, 2)
\end{lstlisting}
